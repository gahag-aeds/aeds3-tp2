\documentclass{article}

\usepackage[margin=1in]{geometry}

\usepackage{algorithm}
\usepackage{amsmath}
\usepackage{amssymb}
\usepackage{booktabs}
\usepackage{calc}
\usepackage{caption}
\usepackage{graphicx}
\usepackage{indentfirst}
\usepackage{lmodern}
\usepackage{multirow,multicol}
\usepackage{newfloat}
\usepackage{relsize}
\usepackage{siunitx}
\usepackage{svg}
\usepackage[super,square]{natbib}
\usepackage[noend]{algpseudocode}
\usepackage[portuguese]{babel}
\usepackage{tikz, tikz-qtree}
\usepackage[utf8]{inputenc}
\usepackage[T1]{fontenc}
\usepackage[bottom]{footmisc}
\usepackage[normalem]{ulem}
\usepackage[hyphens]{url}
\usepackage[hidelinks]{hyperref}


\setlength{\parskip}{5pt}%

\DeclareFloatingEnvironment[fileext=lod]{diagram}
\captionsetup[diagram]{labelformat=empty}
%\captionsetup[figure]{labelformat=empty}

\makeatletter
\renewcommand{\ALG@name}{Algoritmo}
\makeatother



\begin{document}

\begin{titlepage}
  \centering
  
  \vfill{
    \bfseries\Huge
    Universidade Federal de Minas Gerais\\[5pt]
    \bfseries\Large
    Bacharel em Sistemas de Informação \\
    Algoritmos e Estruturas de Dados 3\\
  }
  
  \vfill
  
  \includegraphics[width=13cm]{images/ufmg_logo.jpg}
  
  \vfill{
    \bfseries\Large
    Trabalho Prático 2\\
    Novembro 2017\\
  }
  \vfill{
    \bfseries\large
    Gabriel Silva Bastos\\[5pt]
    Matrícula: 2016058204
  }
\end{titlepage}


\section{Introdução}
A startup de Nubby foi contratada pela Uaibucks para calcular a melhor forma de implantar filiais em determinada cidade. A Uaibucks deseja implantar suas filiais em esquinas, mas com a restrição de não implantar filiais em esquinas vizinhas para evitar auto-concorrência. Além disso, a Uaibucks possui dados da demanda esperada de cada esquina candidata, e deseja maximizar a demanda total no plano de implantação.


\section{Visão Geral da Solução}
O plano de implantação das filiais pode ser modelado como um problema de grafos. No grafo, cada esquina é modelada como um vértice, e cada vértice possui como peso a demanda associada. Há uma aresta entre dois vértices caso as esquinas correspondentes sejam vizinhas. O grafo não é direcionado, pois a relação de vizinhança é comutativa. As arestas não possuem peso. Portanto, o grafo é simples.\footnote{\url{http://mathworld.wolfram.com/SimpleGraph.html}}

No grafo, como cada vértice representa uma esquina, desejamos obter um conjunto de vértices não vizinhos que maximize a demanda total. Para tal, exploramos o conceito de conjunto máximo independente.\footnote{\url{https://en.wikipedia.org/wiki/Maximal_independent_set}} Pela definição, um conjunto máximo independente é um conjunto que não é subconjunto de nenhum outro conjunto independente. Em outras palavras, não há vértice fora do conjunto que, ao ser adicionado, mantenha a propriedade de Independência\footnote{\url{https://en.wikipedia.org/wiki/Independent_set_(graph_theory)}} do conjunto. Considerando que a demanda mínima para uma esquina é zero, um conjunto máximo independente possui demanda total sempre maior ou igual à de seus subconjuntos.

Portanto, para maximizar a demanda, selecionamos dentre todos os conjuntos máximos independentes o conjunto que gera a maior demanda total.

\subsection{NP-Completo}
O problema do conjunto máximo independente pode ser definido como um problema de decisão:
\[ \text{O grafo $G$ possui algum conjunto máximo independente de tamanho $\geq k$?} \]

Partindo do fato que o problema da clique máxima é NP-Completo\footnote{\url{https://en.wikipedia.org/wiki/Clique_problem\#NP-completeness}}, podemos provar que o problema do conjunto máximo independente é NP-Completo.
\begin{equation} \label{eq:cmi-clique}
  \text{Se $S$ é um conjunto máximo independente em um grafo $G$, então $S$ é uma clique máxima no grafo $G'$.}\footnote{\url{https://en.wikipedia.org/wiki/Maximal_independent_set\#Related_vertex_sets}}
\end{equation}
\\[-5pt]
Tal relação permite uma transformação polinomial de uma clique máxima para um conjunto máximo \mbox{independente}, pois o grafo complemento pode ser obtido em tempo polinomial.
\[ \text{Conjunto máximo independente} \propto \text{Clique máxima} \]
Portanto, o conjunto máximo independente é pelo menos tão difícil quanto a clique máxima. \\
Sendo a clique máxima NP-Completo, o conjunto máximo independente é NP-Completo.

\subsection{Entrada}
A entrada consiste de um número $n$ de vértices, e um número $m$ de arestas. Em seguida, os pesos de cada vértice e, logo após, os pares que formam cada aresta.

\subsection{Algoritmo exato}
Há um algoritmo bem conhecido para listar cliques máximas em um grafo: algoritmo de Bron-Kerbosch.\cite{Bron:1973:AFC:362342.362367} Tal algoritmo foi utilizado em conjunto com a propriedade~\eqref{eq:cmi-clique} para listar os conjuntos máximos \mbox{independentes}, e destes selecionar o que gera a maior soma dos pesos dos vértices. A versão com pivoteamento descrita por Tomita \textit{et al}.\cite{TOMITA200628} foi implementada para melhor performance:
\begin{algorithm}
  \caption{Bron-Kerbosch com pivoteamento}
  \begin{algorithmic}[1]
    \Procedure{Bron-Kerbosch}{$G$}
      \State max $\gets \varnothing$
      \State\Call{recursion}{$\varnothing$, $V(G)$, $\varnothing$} \Comment{$V(G)$ is the vertex set of $G$.}
      \State\Return max
    \EndProcedure
    \vspace{5pt}
    \Procedure{recursion}{$R$, $P$, $X$}
      \If {$P \cup X = \varnothing$} \Comment{$R$ is a maximal clique.}
        \If {weight$(R)$ > weight(max)}
          \State max $\gets R$
        \EndIf
      \EndIf
      \vspace{3pt}
      \State $u \gets$ choose one vertex in $P \cup X$ maximizing $|P \cap \Gamma(u)|$ \Comment{$\Gamma(x)$ is the neighbor set of $x$.}
      \For{\textbf{each} vertex $v$ \textbf{in} $P \setminus \Gamma(u)$}
        \State\Call{recursion}{$R \cup \{v\}$, $P \cap \Gamma(v)$, $X \cap \Gamma(v)$}
        \State $P \gets P \setminus \{v\}$
        \State $X \gets X \cup \{v\}$
      \EndFor
    \EndProcedure
  \end{algorithmic}
\end{algorithm}

Para representar os conjuntos de vértices, considerando que $30$ foi a ordem máxima especificada para o grafo, um inteiro de largura fixa de 32 \textit{bits} seria suficiente. Cada \textit{bit} corresponde à presença do vértice no grupo. Tal estratégia torna rápidas as operações necessárias para o algortimo, como união, interseção e subtração de conjuntos.
\begin{align*}
  0 \hdots 0 \> 00011010_2 &\equiv \{1, 3, 4\} \\
  0 \hdots 1 \> 00100101_2 &\equiv \{0, 2, 5, 8\} \\
  0 \hdots 0 \> 11111111_2 &\equiv \{0, 1, 2, 3, 4, 5, 6, 7\}
\end{align*}
Porém, após realizar a análise experimental, foi constatado que o programa executava em um centésimo de segundo para entradas de tamanho $30$. Portanto, para permitir experimentos mais conclusivos, inteiros com largura de 64 \textit{bits} foram utilizados, permitindo entradas com até 64 vértices.

Devido à necessidade recorrente de se obter o conjunto de vizinhos de um vértice no algoritmo, o grafo foi representado de forma que essa operação tenha complexidade $\mathcal{O}(1)$. Para cada vértice, ao invés de uma lista de adjacentes, é armazenado um único inteiro que representa o conjunto dos adjacentes. Tal estratégia requer uma implementação mais elaborada para ler a entrada, mas apresentou uma grande melhora na performance do algoritmo.


\pagebreak


\subsection{Heurística}
Para a heurística, a entrada foi interpretada como uma construção incremental do grafo. Desta forma, antes da leitura das arestas, o grafo é considerado vazio e o conjunto independente possui todos os vértices. Cada aresta lida da entrada é adicionada ao grafo, e possivelmente resulta na remoção de um vértice do conjunto independente. Assim, sempre quando dois vértices no conjunto independente são conectados por uma aresta, um destes é removido do conjunto para manter a propriedade de independência. Ao final da leitura de todas as arestas, o conjunto é independente no grafo.

Este algoritmo é muito simples, mas oferece a vantagem de ser muito rápido por operar com custo baixo logo na leitura de cada aresta, que já é necessária devido ao formato da entrada.

Não há nenhuma garantia de que o conjunto independente gerado será próximo máximo, nem que será próximo do conjunto que gera a maior soma dos pesos. Outro detalhe a ser notado é que o algoritmo é sensível à ordem das arestas na entrada. Portanto, diferentes entradas descrevendo o mesmo grafo podem gerar saídas muito diferentes.

\section{Análise de Complexidade}

\subsection{Algoritmo exato}
\subsubsection{Espacial}
Para cada vértice, um inteiro representando o peso e um inteiro de 64 \textit{bits} representando o conjunto de seus adjacentes são armazenados. Portanto, a complexidade espacial é $\Theta(|V| + |V|)$ = $\Theta(|V|)$.
\subsubsection{Temporal}
De acordo com Tomita \textit{et al}.\cite{TOMITA200628}, a complexidade do algoritmo de Bron-Kerbosch com a técnica de \mbox{pivoteamento} possui complexidade $\mathcal{O}(3^{|V|/3})$ no pior caso. Esta complexidade é ótima, pois segundo Moon e Moser\cite{Moon1965}, um grafo com $n$ vértices possui no máximo $3^{n/3}$ cliques máximas.

\subsection{Heurística}
\subsubsection{Espacial}
Para cada vértice, são armazenados um inteiro representando o peso e um \textit{byte} indicando a presença do vértice no conjunto independente. Portanto, a complexidade espacial é $\Theta(|V| + |V|)$ = $\Theta(|V|)$.
\subsubsection{Temporal}
Devido à construção incremental do grafo, inicialmente este não possui arestas, então o conjunto máximo independente possui todos os vértices. Antes da leitura das arestas, é calculada a soma dos pesos de todos os vértices, com complexidade $\Theta(|V|)$.

Para cada aresta na entrada, uma aresta pode ser removida do conjunto independente. Tal remoção apresenta complexidade $\Theta(1)$, devido à representação do conjunto independente como um vetor de \textit{bytes}. Portanto, a complexidade desta operação é $\Theta(|A|)$.

Complexidade final: $\Theta(|V| + |A|)$.


\pagebreak


\section{Análise Experimental}
A análise experimental das implementações é mostrada pelas figuras. Para realizar os experimentos, diversas entradas foram construídas. Para medir o tempo de execução do código, foi utilizada a informação de uso de recursos fornecida pelo sistema operacional linux.\footnote{\url{https://en.wikipedia.org/wiki/Procfs}} A razão inicial é verificar, de forma geral, o comportamento do algoritmo para casos genéricos.
\begin{figure}[h]
  \centering
  \includesvg[width=0.49\linewidth]{images/exact-time}
  \includesvg[width=0.49\linewidth]{images/exact-space}
  \caption{Ensaio do algoritmo exato.}
  \label{fig:experimental-exato}
\end{figure}
\vspace{-10pt}
\begin{figure}[h]
  \centering
  \includesvg[width=0.49\linewidth]{images/heuristic-time}
  \includesvg[width=0.49\linewidth]{images/heuristic-space}
  \caption{Ensaio da heurística.}
\end{figure} \\
É interessante observar na \autoref{fig:experimental-exato} que o comportamento exponencial do algoritmo exato é notável apenas a partir das entradas de tamanho 50. Isso se deve às estratégias utilizadas para tornar rápidas as operações intrínsecas do algoritmo.
\\[5pt]
Também é notável que a heurística se mostra extremamente rápida, mesmo para entradas muito grandes.


\pagebreak


\subsection{Comparação das soluções}
Para avaliar a qualidade da heurística, uma comparação dos resultados produzidos a partir das entradas \textit{toys} fornecidas pelos monitores foi realizada.
\begin{table}[h]
  \centering
  \begin{tabular}{
      c
      S[ table-number-alignment = center, table-figures-decimal = 0 ]
      S[ table-number-alignment = center, table-figures-decimal = 0 ]
      S[ table-number-alignment = center, table-figures-decimal = 0 ]
    }
    \toprule
    \multicolumn{1}{c}{} & \multicolumn{3}{c}{Demanda total} \\
    \cmidrule{2-4}
    \multicolumn{1}{c}{Toy} & {Algoritmo exato} & {Heurística} & \multicolumn{1}{c}{Degradação} \\ \midrule
     3 & 170 & 170 & { 0\,\%} \\
     7 &  30 &  30 & { 0\,\%} \\
     6 & 139 & 125 & {10\,\%} \\
     5 & 304 & 250 & {18\,\%} \\
     8 & 265 & 199 & {25\,\%} \\
     9 & 312 & 207 & {34\,\%} \\
    10 & 198 & 124 & {37\,\%} \\
     1 & 118 &  57 & {52\,\%} \\
     4 &  10 &   4 & {60\,\%} \\
     2 & 168 &  57 & {66\,\%} \\
    \bottomrule
  \end{tabular}
  \caption{Comparação entre a otimalidade das soluções}
\end{table} \\
É notável que metade dos resultados apresentou degradação inferior a 30\%. Tal resultado indica uma boa aproximação produzida pela heurística.

\section{Conclusão}
Grafos são uma forma elaborada de modelar vários problemas, e esta estrutura de dados apresenta toda uma classe de algoritmos. É importante saber como modelar os problemas em grafos, e quais algoritmos são necessários para extrair cada tipo de informação necessária para o problema.

Problemas NP-Completo estão presentes na nossa realidade, e precisamos de ferramentas eficientes para resolvê-los. Além da formalidade necessária para provar que um problema é NP-Completo, desenvolvemos diferentes formas de abordar o problema.

Algoritmos exatos produzem respostas ideais, e são úteis não só para obter tais respostas, mas também para fornecer uma base de comparação para outras soluções. Porém, estes algoritmos são viáveis apenas para entradas pequenas, devido à sua alta complexidade.

Algoritmos aproximados são interessantes quando existe a necessidade de um certo nível de correção da resposta, porém pode ser complexo o desenvolvimento do algortimo e da prova da sua razão de aproximação.

Heurísticas são especialmente interessantes quando precisamos de uma resposta para o problema, não necessariamente próxima da ótima, mas de forma muito rápida. É interessante como heurísticas simples podem apresentar bons resultados em complexidade muito inferior à do algoritmo exato.


\pagebreak

\bibliography{bib/tomita,bib/bron-kerbosch,bib/moon-moser}
\bibliographystyle{plain}

\end{document}
